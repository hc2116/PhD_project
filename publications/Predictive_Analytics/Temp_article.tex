%%
%% This is file `sample-acmsmall.tex',
%% generated with the docstrip utility.
%%
%% The original source files were:
%%
%% samples.dtx  (with options: `acmsmall')
%% 
%% IMPORTANT NOTICE:
%% 
%% For the copyright see the source file.
%% 
%% Any modified versions of this file must be renamed
%% with new filenames distinct from sample-acmsmall.tex.
%% 
%% For distribution of the original source see the terms
%% for copying and modification in the file samples.dtx.
%% 
%% This generated file may be distributed as long as the
%% original source files, as listed above, are part of the
%% same distribution. (The sources need not necessarily be
%% in the same archive or directory.)
%%
%% The first command in your LaTeX source must be the \documentclass command.
\documentclass[acmsmall]{acmart}

%%
%% \BibTeX command to typeset BibTeX logo in the docs
\AtBeginDocument{%
  \providecommand\BibTeX{{%
    \normalfont B\kern-0.5em{\scshape i\kern-0.25em b}\kern-0.8em\TeX}}}

%% Rights management information.  This information is sent to you
%% when you complete the rights form.  These commands have SAMPLE
%% values in them; it is your responsibility as an author to replace
%% the commands and values with those provided to you when you
%% complete the rights form.
\setcopyright{acmcopyright}
\copyrightyear{2018}
\acmYear{2018}
\acmDOI{10.1145/1122445.1122456}


%%
%% These commands are for a JOURNAL article.
\acmJournal{JACM}
\acmVolume{37}
\acmNumber{4}
\acmArticle{111}
\acmMonth{8}

%%
%% Submission ID.
%% Use this when submitting an article to a sponsored event. You'll
%% receive a unique submission ID from the organizers
%% of the event, and this ID should be used as the parameter to this command.
%%\acmSubmissionID{123-A56-BU3}

%%
%% The majority of ACM publications use numbered citations and
%% references.  The command \citestyle{authoryear} switches to the
%% "author year" style.
%%
%% If you are preparing content for an event
%% sponsored by ACM SIGGRAPH, you must use the "author year" style of
%% citations and references.
%% Uncommenting
%% the next command will enable that style.
%%\citestyle{acmauthoryear}

%%
%% end of the preamble, start of the body of the document source.
\begin{document}

%%
%% The "title" command has an optional parameter,
%% allowing the author to define a "short title" to be used in page headers.
\title{The Name of the Title is Hope}

%%
%% The "author" command and its associated commands are used to define
%% the authors and their affiliations.
%% Of note is the shared affiliation of the first two authors, and the
%% "authornote" and "authornotemark" commands
%% used to denote shared contribution to the research.
\author{Ben Trovato}
\authornote{Both authors contributed equally to this research.}
\email{trovato@corporation.com}
\orcid{1234-5678-9012}
\author{G.K.M. Tobin}
\authornotemark[1]
\email{webmaster@marysville-ohio.com}
\affiliation{%
  \institution{Institute for Clarity in Documentation}
  \streetaddress{P.O. Box 1212}
  \city{Dublin}
  \state{Ohio}
  \postcode{43017-6221}
}

\author{Lars Th{\o}rv{\"a}ld}
\affiliation{%
  \institution{The Th{\o}rv{\"a}ld Group}
  \streetaddress{1 Th{\o}rv{\"a}ld Circle}
  \city{Hekla}
  \country{Iceland}}
\email{larst@affiliation.org}


%%
%% By default, the full list of authors will be used in the page
%% headers. Often, this list is too long, and will overlap
%% other information printed in the page headers. This command allows
%% the author to define a more concise list
%% of authors' names for this purpose.
\renewcommand{\shortauthors}{Trovato and Tobin, et al.}

%%
%% The abstract is a short summary of the work to be presented in the
%% article.
\begin{abstract}
  A clear and well-documented \LaTeX\ document is presented as an
  article formatted for publication by ACM in a conference proceedings
  or journal publication. Based on the ``acmart'' document class, this
  article presents and explains many of the common variations, as well
  as many of the formatting elements an author may use in the
  preparation of the documentation of their work.
\end{abstract}

%%
%% The code below is generated by the tool at http://dl.acm.org/ccs.cfm.
%% Please copy and paste the code instead of the example below.
%%
\begin{CCSXML}
<ccs2012>
 <concept>
  <concept_id>10010520.10010553.10010562</concept_id>
  <concept_desc>Computer systems organization~Embedded systems</concept_desc>
  <concept_significance>500</concept_significance>
 </concept>
 <concept>
  <concept_id>10010520.10010575.10010755</concept_id>
  <concept_desc>Computer systems organization~Redundancy</concept_desc>
  <concept_significance>300</concept_significance>
 </concept>
 <concept>
  <concept_id>10010520.10010553.10010554</concept_id>
  <concept_desc>Computer systems organization~Robotics</concept_desc>
  <concept_significance>100</concept_significance>
 </concept>
 <concept>
  <concept_id>10003033.10003083.10003095</concept_id>
  <concept_desc>Networks~Network reliability</concept_desc>
  <concept_significance>100</concept_significance>
 </concept>
</ccs2012>
\end{CCSXML}

\ccsdesc[500]{Computer systems organization~Embedded systems}
\ccsdesc[300]{Computer systems organization~Redundancy}
\ccsdesc{Computer systems organization~Robotics}
\ccsdesc[100]{Networks~Network reliability}

%%
%% Keywords. The author(s) should pick words that accurately describe
%% the work being presented. Separate the keywords with commas.
\keywords{datasets, neural networks, gaze detection, text tagging}


%%
%% This command processes the author and affiliation and title
%% information and builds the first part of the formatted document.
\maketitle

\section{Introduction}
ACM's consolidated article template, introduced in 2017, provides a
consistent \LaTeX\ style for use across ACM publications, and
incorporates accessibility and metadata-extraction functionality
necessary for future Digital Library endeavors. Numerous ACM and
SIG-specific \LaTeX\ templates have been examined, and their unique
features incorporated into this single new template.

If you are new to publishing with ACM, this document is a valuable
guide to the process of preparing your work for publication. If you
have published with ACM before, this document provides insight and
instruction into more recent changes to the article template.

The ``\verb|acmart|'' document class can be used to prepare articles
for any ACM publication --- conference or journal, and for any stage
of publication, from review to final ``camera-ready'' copy, to the
author's own version, with {\itshape very} few changes to the source.


\cite{tavallaee2009detailed}


\section{Review Procedure}
This section describes the inclusion criterions for the review, the method used to identify the relevant literature, the elements extracted from the literature and process of extracting this information.


\subsection{Research Questions}

\paragraph{Research Question 1:} What predictive analytics metrics have been proposed in the literature?

\paragraph{Research Question 2:} ....?

\subsection{Inclusion Criterion}
As noted above, the issue predicting adversarial cyber operations involves both straightforward cases were a known threat is to be recognized and more or less impossible cases where previously unknown attacks performed by professionals should be predicted. The panel and the literature search aimed at focusing on research in-between these two extremes. It was also recognized that much of the extant research on cyber security have some relation to prediction. For example, when the relationship between cyber variables are investigated in terms of statistical relationships this offers potential input to prediction models. However, as this review is mostly concerned with existing prediction models, and not surveying potential building blocks of new ones, such research was excluded. Thus, the primary inclusion criteria was that the papers should describe a model explicitly developed for predicting adversarial cyber operations. 
In addition, the predictions produced by the model should be able to support a decision maker within an organization to secure their cyber environment. This meant that papers listing attacks that were possible to perform against the cyber environment were excluded they did not make statements about the likelihood that the attacks would be performed within a certain time frame. For instance, most papers on attack graphs, which rarely state how likely attacks are to happen, were excluded. Similarly, papers on models producing generic predictions, such as “more attacks of type X will happen next year”, were excluded unless it was possible to tune the predictions based on information related to a specific organization or cyber environment. Finally, some papers were excluded because they were too abstract and superficial.  For example, papers stating basic ideas about how threat intelligence could foster prediction, without detailing what this threat intelligence comprise of, were excluded.

\subsection{Literature Search}
Some initial attempt with the definition of search terms that could be used to target literature related to predicting adversarial cyber operations. However, it was soon acknowledged that this literature was too diverse and scattered to be identified by a straightforward search in literature databases. Fortunately, the panel members participating in the review process was diverse set of researchers, with backgrounds in cyber security, mathematics, forensics, combat assessments, visualisation, and military intelligence . This diversity helped to identify a fifteen topics that were related. These topics included, among others, “the cyber OODA-loop”, “cyber attack profiling”, “data fusion approaches”, “risk and incident management”, “situation description methods”, “the relationship between capabilities and actions”, “unknown vulnerability detection”, “attack probability indicators”, and “multi-step attack models”. 
Panel members were assigned to topics they had previous knowledge of and searched for literature within this topic that could match the search criterion. These searches typically started with searches in scholarly databases to identify research papers and internet searches to find “grey literature” (e.g. technical reports). Such searches were complemented by inspection of references used in papers considered relevant. The search process hardly can be described as structured. However, the range of competences and background among the panel members ensured that it covered a wide range of topics related to cyber security. A database of 36 papers had been identified and passed the inclusion criterion after the group screened the paper’s title and abstract. After review of the full text XX papers were found to meet the inclusion criterion. These XX can not be said to represent an all available papers related to prediction of adversarial cyber operations, but it is the group’s opinion that they are likely to indicate the state of research in this area. 



\subsection{Information Extraction}
The literature addressed the prediction problem from a number of different angles, and at different levels of abstraction. At first, an input-output-perspective attempted to broadly characterize the data the models used. The model of STIX [REF] was used for this purpose. However, the analysis soon showed that it was that most of the models used more or less the same data objects in STIX, namely: vulnerability-information, attack patterns, indicators, intrusion sets, and other observed data. In addition, it turned out to be non-trivial to classify the data used in a reliable manner, partly because of the different levels of abstraction used in the papers. Instead, the following information was extracted to characterize the models:
\begin{enumerate}
\item If a particular formalism was used or proposed.
\item Data used as input for the prediction model.
\item The data produced as output by the prediction model.
\item The scalability of the solution or implementation. 
To further characterize the models it was extracted how they handled the following issues:
\item Adversaries attempting to tampering with/fooling the prediction method.
\item The time it takes to make the prediction and timing issues.
\item Availability of data needed for analysis or model construction.
\item Assumptions concerning knowledge of system vulnerabilities and attacks.
Furthermore, information was extracted to characterize the maturity of the research in terms by assessing:
\item If the model had been implemented in prototype and what Technology Readiness Level the model was on.
\item If the model’s its usefulness for was demonstrated, e.g. in a case study.
\item Tests or other evaluations of accuracy of the prediction model.
These nine information elements was extracted as quotes and descriptive summaries of descriptions provided in the reviewed 
\end{enumerate}



%%
%% The next two lines define the bibliography style to be used, and
%% the bibliography file.
\bibliographystyle{ACM-Reference-Format}
\bibliography{PredAna}


\end{document}
\endinput
%%
%% End of file `sample-acmsmall.tex'.
