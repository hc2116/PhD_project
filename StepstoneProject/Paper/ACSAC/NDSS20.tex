 \documentclass[conference]{IEEEtran}\usepackage[]{graphicx}\usepackage[]{color}
%% maxwidth is the original width if it is less than linewidth
%% otherwise use linewidth (to make sure the graphics do not exceed the margin)
\makeatletter
\def\maxwidth{ %
  \ifdim\Gin@nat@width>\linewidth
    \linewidth
  \else
    \Gin@nat@width
  \fi
}
\makeatother

\definecolor{fgcolor}{rgb}{0.345, 0.345, 0.345}
\newcommand{\hlnum}[1]{\textcolor[rgb]{0.686,0.059,0.569}{#1}}%
\newcommand{\hlstr}[1]{\textcolor[rgb]{0.192,0.494,0.8}{#1}}%
\newcommand{\hlcom}[1]{\textcolor[rgb]{0.678,0.584,0.686}{\textit{#1}}}%
\newcommand{\hlopt}[1]{\textcolor[rgb]{0,0,0}{#1}}%
\newcommand{\hlstd}[1]{\textcolor[rgb]{0.345,0.345,0.345}{#1}}%
\newcommand{\hlkwa}[1]{\textcolor[rgb]{0.161,0.373,0.58}{\textbf{#1}}}%
\newcommand{\hlkwb}[1]{\textcolor[rgb]{0.69,0.353,0.396}{#1}}%
\newcommand{\hlkwc}[1]{\textcolor[rgb]{0.333,0.667,0.333}{#1}}%
\newcommand{\hlkwd}[1]{\textcolor[rgb]{0.737,0.353,0.396}{\textbf{#1}}}%
\let\hlipl\hlkwb

\usepackage{framed}
\makeatletter
\newenvironment{kframe}{%
 \def\at@end@of@kframe{}%
 \ifinner\ifhmode%
  \def\at@end@of@kframe{\end{minipage}}%
  \begin{minipage}{\columnwidth}%
 \fi\fi%
 \def\FrameCommand##1{\hskip\@totalleftmargin \hskip-\fboxsep
 \colorbox{shadecolor}{##1}\hskip-\fboxsep
     % There is no \\@totalrightmargin, so:
     \hskip-\linewidth \hskip-\@totalleftmargin \hskip\columnwidth}%
 \MakeFramed {\advance\hsize-\width
   \@totalleftmargin\z@ \linewidth\hsize
   \@setminipage}}%
 {\par\unskip\endMakeFramed%
 \at@end@of@kframe}
\makeatother

\definecolor{shadecolor}{rgb}{.97, .97, .97}
\definecolor{messagecolor}{rgb}{0, 0, 0}
\definecolor{warningcolor}{rgb}{1, 0, 1}
\definecolor{errorcolor}{rgb}{1, 0, 0}
\newenvironment{knitrout}{}{} % an empty environment to be redefined in TeX

\usepackage{alltt}
\pagestyle{plain}
\usepackage[cmex10]{amsmath}
%\usepackage{caption} 
%\captionsetup[table]{skip=3pt}

\hyphenation{op-tical net-works semi-conduc-tor}
\IfFileExists{upquote.sty}{\usepackage{upquote}}{}
\begin{document}

\title{Evading passiv stepping stone detection by using enough chaff}
%Evaluation of passiv stepping stone detection techniques under chaff and delay}


\author{\IEEEauthorblockN{Michael Shell}
\IEEEauthorblockA{Georgia Institute of Technology\\
someemail@somedomain.com}
\and
\IEEEauthorblockN{Homer Simpson}
\IEEEauthorblockA{Twentieth Century Fox\\
homer@thesimpsons.com}
\and
\IEEEauthorblockN{James Kirk\\ and Montgomery Scott}
\IEEEauthorblockA{Starfleet Academy\\
someemail@somedomain.com}}


\IEEEoverridecommandlockouts
\makeatletter\def\@IEEEpubidpullup{6.5\baselineskip}\makeatother
\IEEEpubid{\parbox{\columnwidth}{
    Network and Distributed Systems Security (NDSS) Symposium 2020\\
    23-26 February 2020, San Diego, CA, USA\\
    ISBN 1-891562-61-4\\
    https://dx.doi.org/10.14722/ndss.2020.23xxx\\
    www.ndss-symposium.org
}
\hspace{\columnsep}\makebox[\columnwidth]{}}


\maketitle          

\begin{abstract}

Bla
\end{abstract}


\section{Introduction}

%\textcolor{red}{Network attackers frequently use a chain of compromised intermediate nodes to attack a target machine and maintain anonymity. This chain of nodes between the attacker and the target is calleda stepping stone chain. }


Malicious actors on the Internet frequently use chains of compromised hosts to relay their attack, in order to obtain access to restricted resources and to reduce the chance of being detected. These hosts, called \textbf{stepping-stones}, are used by the attacker as relay machines, to which they maintain access using tools such as SSH or telnet \textcolor{red}{reference}. 

Accessing a server via multiple relayed TCP connections can make it harder to tell the intruder's geographical location, and enables attackers to hide behind a long interactive stepping-stone chain. Furthermore, it is often required to relay an attack via privileged hosts in a network that have access to restricted resources. 

However, detecting that a host is used in a stepping-stone chain is a clear indication of malicious behaviour. If a stepping-stone intrusion can be detected during the attack stage, the connection can be terminated to interrupt the attack. Stepping-stone detection (SSD) primarily looks at network traffic, with many approaches aiming to identify potential correlation between two connections going from or to a particular host. To hide correlation between relayed packets, intruders can impose delays on packet transfer and inject additional \textit{chaff} packets into the connections.

There are a number of approaches to detect stepping-stones, with the earliest one having been proposed by Zhang and Paxson in 2000 \textcolor{red}{citation needed?}. However like many intrusion attacks, stepping-stones are rare and there exists no public data representing real stepping-stone behaviour, and researchers have to rely on synthetic data. Some attempts have been made to create publicly available stepping-stone testbeds, but most researchers evaluate their SSD methods on self-provided data. The underlying generating implementations often differ significantly, which makes their direct comparison of the achieved results impossible. Additionally, we find that implemented evasive behaviours are often too simplistic and thus increase detection rates. 

In this work, we provide an independent framework to generate data representing stepping-stone behaviour ....
We use this data to provide an in-depth evaluation of current passive SSD methods and compare their results. Furthermore, we show that by inserting enough chaff perturbations in the right form, an intruder can evade all current SSD methods successfully


\section{Background}

When a person logs into one computer and from there logs into another computer and so on, we refer to the sequence of logins as a connection chain[18]. 
In an interactive stepping-stone attack, an attacker located at the origin host, which we call \textit{host O} sends commands to and awaits their response from a target, \textit{host T}, typically by using terminal emulation programs like TelNet or SSH. These commands and responses are proxied via a chain of one or more intermediary hosts, the stepping-stones, which we call \textit{host} $S_1$, \dots, $S_N$, such as depicted in Fig. \textcolor{red}{insert image}. 

Stepping-stone detection (SSD) is a process of observing all incoming and outgoing connections in a network and determining which ones are parts of a connection chain. %This problem is closely related to the problem of tracing intruders through the Internet by following the connection chain. 
SSD typically either aims at classifying a host as a stepping stone if two connections involving this host appear to be correlated, or at classifying a connection as part of a chain if its behaviour differs from regular connections. In addition, several SSD methods aim to estimate the overall length of the stepping-stone chain to help tracing the intruder by following the connection chain. 
A traffic collection sensor is typically placed in the vicinity of the examined host to provide the necessary data. 

\textcolor{red}{insert citation} identify another form of stepping-stones called \textit{store-and-forward}, which transfer data within files in a non-interactive manner. Though harder to detect than interactive connections, this procedure limits the attackers ability to explore the target, which is why SSD research has been primarily concerned with interactive stepping-stones.

\subsection{Packet delays}

\subsection{Chaff perturbations}



\section{Related work}\label{Sec:Relatedwork}

%\subsubsection{Evasion tactics}

%Padhye et al. \cite{padhye2010evading} have 

\subsection{Datasets}


In 2006, Xin et al. \cite{xin2006testbed} developed a standard test bed for stepping-stone detection, called \textit{SST}. 
The main objectives in the development of SST were to enable a reproducible evaluation of stepping stone chain detection algorithms with easy configuration, management and operation. The tool allows for an arbitrary number of intermediate hosts and generates scripts to mimic interactive SSH and TelNet connections. 
To insert time delays and chaff perturbations, the authors modified the OpenSSH protocol on the intermediary hosts. Delays can be drawn from a uniform distribution while chaff can be drawn from a Poisson or Pareto distribution. To our knowledge, SST has only been used for evaluation by Zhang et al. \cite{zhang2005stepping}, and is not available anymore. \textcolor{red}{say something why evasion not sufficiently implemented}

Another approach to use publicly available data comes from Houmansadr et al. \textcolor{red}{insert citations}. The authors use the well-known CAIDA 2016 anonymized data traces \textcolor{red}{insert citation} to evaulate different watermarking methods for SSD. To simulate stepping stone behaviour, packet delays and drops are imposed retroactively on selected connections using Laplace and Bernoulli distributions with different rates. While this procedure seems sufficient for the evaluation of watermarking methods, it falls short on simulating an actual connection chain and leaves out chaff perturbations. 


Wang et al. \cite{wang2018research} recently conducted an extensive survey of  stepping stone intrusion detection. The authors group methods according to the respective methodology into 
\begin{itemize}
\item content-thumbprint,
\item time-thumbprint,
\item packet counting,
\item random-walk-based,
\item cross-over packet-based,
\item watermarking,
\item network-based,
\item and software-defined-networking-based, 
\end{itemize} 
but do not cover \textcolor{red}{graph-based methods} such as \cite{gamarra2018analysis} %or \cite{apruzzese2017detection}, 
or anomaly-based methods such as \cite{di2011detecting}, which are increasing in popularity recently.
The authors then proceed to explain the different methods and highlight their benefits and shortcomings. The authors discuss open problems, but do not provide a comparison of detection rates. 

Shullich et al. \cite{shullich2011survey} in 2011 also conducted a survey on stepping stone intrusion detection. The authors perform a similar grouping of methods, but also discuss related work in evasion tactics and test frameworks. The authors furthermore give an outlook on areas for future research, such as hacker motivation, the cardinality problem when correlating connection pairs, the difficulty of tracing back chains through firewalls, the lack of real-world data examples, or detection in covert channels or protocols such as UDP.  

\cite{almulhem2007survey}

Stepping Stone Detection Techniques: Classification and State-of-the-Art, bad though

Metrics: A Study on the Performance Metrics forEvaluating Stepping Stone Detection (SSD)
Stepping Stone Detection: Measuring the SSD Capability


\section{Dataset creation}



\subsection{Simulating stepping stones with SSH-tunnels and Docker}

insert figure with one and with three stepping stones



SSH tunnel on respective port on the starting point of the chain, tunnels to port on the next point in the chain. Finally, Refer to figure. 

\subsubsection{Adding network congestion}

Docker communication takes place over virtual bridge networks, so the throughput is far higher and more reliable than in real-world networks. %This level of speed and consistency is worrying for our purposes as packet timings will be largely identical on repeated runs of a scenario and any collected data could be overly homogeneous.
To retard the quality of the Docker network to realistic levels, we rely on the emulation tools Netem. Netem \textcolor{red}{add reference} is a Linux command line tool that allows users to artificially simulate network conditions such as high latency, low bandwidth or packet corruption in a flexible manner.
We apply Netem commands to the network interface of each container, which adds correlated delays to incoming and outgoing packets that are drawn from a normal distribution with mean $\mu$, variance $\sigma^2$, and correlation $\rho_1$. We furthermore apply correlated packet loss and corruption drawn from a binomial distribution with probability $p$ and correlation $\rho_2$.

We set the network settings for the starting point and the end point container individually and draw each of the given parameters from a suitable distribution \textcolor{red}{(should I specify which one for each? Seems a bit much...)} before each \textcolor{red}{run} to allow for a good amount of variation in the generated data.

%providing us with the flexibility to set each container's network settings uniquely. 




%This script randomizes the values of each parameter, such as packet drop rate, bandwidth limit, latency, ensuring that every run of a scenario has some degree of network randomization if desired.



\subsubsection{Adding delays}

To add artifical delays to forwarded packets on a stepping stone host for detection evasion, we can again use NetEm. We draw delays for departing packets from a uniform distribution, as suggested by \textcolor{red}{add reference}, covering the interval $[0,\delta_d]$, with no packet correlation. 

\subsubsection{Adding chaff perturbation}

To add chaff packets to the relayed connection, we forward two additional ports through the SSH-tunnel of a stepping stone host. We then use NetCat \textcolor{red}{add reference} to send data to both ports from either direction and collect it at the other side. Figure \textcolor{red}{...} depicts this setup for an individual tunnel. 
The data sent through tunnel $i$ consists of strings with random size $x$ drawn from a Cauchy-distribution with mean $xx_i$, and is sent in intervals of random length $\delta_c$ drawn from an \textcolor{red}{paretonormal} distribution with mean $yy_i$. By adjusting $yy_i$, we can control the amount of chaff sent through a tunnel. 

\subsection{Simulating interactive SSH-traffic}

In order to generate enough data instances representing interactive stepping stone behaviour, we automatised the communication between the start point and the end point of the stepping stone chain. To do so, we generate a script with SSH-commands at the start of each \textcolor{red}{execution} that is passed and run by the \textcolor{red}{starting point} of the chain. The generated script consists of a sequence of ordinary SSH-commands \textcolor{red}{list them here?}, which are drawn randomly from a command catalogue and are each separated by \textit{sleep}-commands for a time $t$ that is drawn each time from a Cauchy-distribution. The average sleep-time is around \textcolor{red}{insert}. The length of the script is reached when the \textit{end}-command is drawn from the catalogue.
%To evaluate the stepping stone detection capabilities of different proposed methods, we created a \textcolor{red}{realistic dataset...}
\textcolor{red}{Insert example}

%For script-based session creation, several measures havebeen taken to make them realistic. First, each session tries tomimic a real user’s action. All users of the scripts are real useraccounts in the clients. The users are divided into four groups:administrators, secretaries, programmers, and researchers. Foreach type of user, there is large command database which con-sists of the common commands and their historic frequency ofexp_version -exit 5.0if {$argc!=1} {send_user "usage: ftp-rfc \[#] \[-index]\n"exit}set file "rfc$argv.Z"set timeout 60spawn ftp ftp.uu.netexpect "Name*:"send "anonymous\r"expect "Password:"send "expect@nist.gov\r"expect "ftp>"send "binary\r"expect "ftp>"send "cd inet/rfc\r"expect "550*ftp>" exit "250*ftp>"send "get $file\r"expect "550*ftp>" exit "200*226*ftp>"closewaitsend_user "\nuncompressing file - wait...\n"exec uncompress $fileFig. 7.  Example script for a TELNET sessionuse [22]. A portion of the command database for programmersare shown in Table I. Each Expect script is associated witha user, and the commands within the script are drawn fromcorresponding database according to their frequency. Thereare two types of commands: atomic and compound. Atomiccommands only need the user type the command line andthe system will parse the command and print the output.Examples are “ls -la”, “pwd”. Compound commands requiremore interaction. A simple example is “mail xxxx”. After theuser enters “mail xxxx” and “Enter” key, the system promptsthe user to enter “subject”, content and “cc” field of the email.Second, the human typing simulation feature of Expect is usedfor all commands of users. Third, more than 50 human actorsfrom four classes were invited to our lab to generate certainstepping stone chains that were recorded by autoExpect.


\subsection{HTTP-interactions}
 
In order to provide an additional, different type of interaction between the \textcolor{red}{starting point} and \textcolor{red}{end point}, we directed HTTP traffic over the stepping stone chain. Here, the starting point hosts Scrapy, a web crawling service \textcolor{red}{insert citation}, that surfs the 1 million most popular website by clicking links on them. The requests are sent over the stepping stone chain to the web. 

This type of traffic is not meant to necessarily represent realistic stepping stone behaviour, but to provide an additional source of interactive traffic that differs substantially from SSH in order to test detection methods from another angle.




%\begin{tabular}{r|cccc|cccc}
%\multicolumn{1}{r|}{ }&\multicolumn{4}{c|}{SSH}&\multicolumn{4}{c}{HTTP}\\
%bla& no pert.&var. delays&var. chaff&delay\&chaff & no pert.&var. delays&var. chaff&delay\&chaff \\
%\end{tabular}

\begin{tabular}{r|cccc}
%\multicolumn{1}{r|}{ }&\multicolumn{4}{c|}{SSH}&\multicolumn{4}{c}{HTTP}\\
SSH 1 node& no pert.&var. delays&var. chaff&delay\&chaff \\
HTTP 1 node& no pert.&var. delays&var. chaff&delay\&chaff \\
SSH 3 node& no pert.&var. delays&var. chaff&delay\&chaff \\
HTTP 1 node& no pert.&var. delays&var. chaff&delay\&chaff \\
\end{tabular}


 
\cite{sommer_outside_2010}

%\subsection{Regular traffic congestion}

%Packet transmission in Docker's virtual network is almost instantaneous and not subject to transmission errors. 
%To emulate realistic traffic as close as possible, we add emulated traffic congestion in the form of packet delivery delays and packet losses. These are implemented between individual hosts using tc-netem.

%Delays are added to incoming and outgoing packets on each host individually, and are drawn from a normal distribution. The mean and standard deviation are drawn before each simulation for each host.


\section{Selected approaches}

Researchers have so far proposed two main approaches: passive monitoring and active perturbation. In the latter

\subsection{Packet-correlation-based approaches}

%Efficient multi-dimensional flow correlation

%\subsubsection{Detecting Connection-Chains: A Data Mining Approach, 2010}


%TPR 100\% and FPR 0\%

\subsubsection{Correlating TCP/IP Packet contexts to detect stepping-stone intrusion 2011}

Yang et al. \cite{yang2011correlating} compare sequences of interarrival times in connection pairs to detect potential stepping-stone behaviour. For that, the context of a packet is defined as the packet interarrival times around that packet. The context of  packets is extracted from each connection, and their respective contextual distance is estimated using the Pearson correlation. Packets with high correlation are defined as `matched', and two connections are classified as relayed if the ratio of matched packets exceeds a threshold. The authors propose to only collect interarrival times from \textit{Echo}-packets instead of \textit{Send}-packets to resist evasion tactics such as chaff and jitter, as the sending of \textit{Echo}-packets is subject to more constraints and less easy to manipulate.

The authors evaluate their results on connection pairs with up to $100\%$ chaff ratio, with the model being able to successfully detect connection relays in all cases. 


\subsection{Neural networks}

Similar to other areas in intrusion detection, researchers have recently begun to train artificial neural networks to identify stepping stones. 

A notable initial example came from \textcolor{red}{citation}, who ...
Performance of neural networks in stepping-stone intrusion detection 2008
however, the authors concluded that the achieved results did not improve existing detection rates. but no good results, better in Neural  networks-based  detection  of stepping-stone  intrusion

The authors present two networks to identify stepping stones. 
The first method uses eight packet variables of an individual packet as input to predict the number of stepping-stones.

The second method is based on sequences of RTTs. For this, a packet matching algorithm is used to compute RTTs, which are then fed as a fixed-length sequence into a feed-forward network to predict the downstream length. The network itself only contains one hidden layer and is relatively small. The results from the packet-based method are however inconclusive, and the RTT-based methods achieves good results only if RTTs are small, i.e. the stepping-stone chain is completely contained within one LAN-network.



A more recent example comes from \textcolor{red}{citaton}, who train a convolutional neural network to identify correlation between two simultaneous connections from the upstream and downstream interarrival times and packet sizes in each connection. The trained network is large, with over 200 input filters and consists of three convolutional and three feed-forward layers. The trained model was initially applied to a dataset of Tor-connections as well, where the authors achieved strong results. 
The authors achieve a $90\%$ detection rate with $0.02 \%$ false positives. 


\subsection{RTT-based approaches}

Another prominent approach to detect stepping stones is based on \textit{Round-trip/times} (RTTs). The RTT of a connection is the time it takes for a packet to be sent to the receiver plus the time it takes for an acknowledgement of that packet to be received. For a normal connection, the measured RTTs should be centered closely around one value. However, since information is relayed over one or more hosts in a stepping stone chain, \textcolor{red}{the assumption for RTT-based detection} is that the responses from different hosts within the chain generate multiple RTTs. Observing multiple RTTs is therefore a clear indication of relaying behaviour.

Yang et al. \cite{yang2015rtt,yang2007mining} and Huang et al. \cite{huang2016detecting,ding2009detecting,huang2007stepping}  both have proposed multiple approaches for estimating and employing RTTs for stepping stone detection. We have selected two papers that depict the \textcolor{red}{state-of-the-art}...

\subsubsection{RTT-based Random Walk Approach to Detect Stepping-Stone Intrusion }


This model by Yang et al. \cite{yang2015rtt} combines packet-counting methods and RTT mining methods to improve detection results from \cite{yang2007mining}. 

A widely-used approach is to compare the number of incoming packets in one connection with the number of outgoing packets in another connection  to determine if the pair represents a stepping stone relay. However, the insertation of chaff can \textcolor{red}{separate} these numbers substantially. 
To resist intruders  evasion,  the authors  propose  to use the  number of  round-trips in  a  connection  to  determine  if  the  connection  is being  relayed.
Packet pairs representing a round-trip for each connection are estimated using a combination of packet matching and clustering, and counted as $N_{in}$ and $N_{out}$. The authors then claim that the value of $N_{in}-N_{out}$ is only bounded if the two connections are relayed.


\subsubsection{Detecting Stepping-Stone Intruders by Identifying Crossover Packets in SSH Connections, 2016}

This method by Huang et al. \cite{huang2016detecting} improves the detection methods proposed by Ding et al. \cite{ding2009detecting}. The authors target specifically relayed interactive SSH communication at the end of a connection chain. They build their detection model on the fact that in a long connection chain, the round-trip time of a packet may be longer than the intervals between two consecutive keystrokes. Normally after sending a request packet, a client will wait for the server response before sending another request. However, TCP/IP allows a client to send a limited number of packets to the server without having to wait for the response. In a long connectin chain, this will result in cross-overs between request and response, which causes the curve of sorted Upstream RTTs to rise more steeply than in a regular connection. A stepping stone is detected if the maximum increase in the curve exceeds a threshold. The authors do not state a universal threshold value and instead suggest a method to estimate the appropriate value for a given setting.
%Detecting Stepping-Stone Intruders by Identifying Crossover Packets in SSH Connections

%.

%.

%RTT-based Random Walk Approach to Detect Stepping-Stone Intrusion 

%Detecting Stepping-Stone Intruders with Long Connection Chains? 2009


\subsection{Anomaly-based approaches}

Since the insertion of time delays and chaff perturbations is so far relatively successful at evading detection, two authors have proposed algorithms to detect these two activities in a connection as deviations from typical TCP-behaviour to indicate of suspicious behaviour. 

\subsection{Detecting Anomalies in Active Insider Stepping Stone Attacks, 2011}

Crescenzo et al. \cite{di2011detecting} have proposed a \textcolor{red}{assembly} of three methods to detect time delays and chaff perturbations in a selected connection. 
\begin{enumerate}
\item The response-time based method is targeted at detecting packet time-delays. It estimates the RTT of a connection, and then flags acknowledgement packets if their response-time exceeds (RTT+$\delta_{RT}$). The authors claim that an attacker would have to impose delays on more than 70\% of all packets to evade this method.
\item The edit-distance based method is targeted at detecting 
\end{enumerate}

\subsection{Detecting Chaff Perturbation on Stepping-Stone Connections, 2011, and Detecting Stepping-Stones under the  Influence of Packet Jittering, 2013}
Huang et al. \cite{huang2011detecting} proposed a method to detect chaff perturbations in individual connections based on their assessment that interarrival times in regular connections tend to follow a Pareto or Lognormal distribution whereas chaffed connections do not. To detect whether a connection contains chaff perturbations, the authors extract packet interarrival times and fit a probability distribution to the data using maximum likelihood estimation. Afterwards, the goodness-of-fit is tested using two statistical tests, which yield a \textit{disagreement-of-fit score}. If this disagreement score exceeds a threshold, which was determined from a set of know regular connections, the connection is seen as being subject to chaff perturbations. The authors generate a small set of chaffed interactive SSH stepping-stone chains, where they achieve a $95\%$ detection rate on connections which are subject to a $50\%$ chaff ratio while retaining zero false positives. For lower chaff ratios, the detection rate decreases significantly. 

A similar approach from the same authors \cite{ding2013detecting} is directed towards the detection of packet jittering. However, instead of estimating a disagreement-of-fit score, the authors use the estimated distribution parameters as input to train a support vector machine. 



\subsection{Behaviour-based approaches}

A different direction in the detection of stepping stones focuses more on the general communication behaviour of selected hosts rather than individual connections. Features include the timely correlation of connecting IP-address on a selected host or unusual paths of simultaneously existing connections within a computer network.

Emulating these features realistically in network traffic datasets is difficult, and there exist little research on how strongly actual attack behaviour influences these features. Since our dataset focuses only on individual connections and corresponding evasion, \textcolor{red}{we are not able to include behaviour-based approaches in our evaluation}.




\subsection{Evaluation methods}

All of the above presented methods classify individual connections or pairs of connections as malicious or benign. True stepping stone connections are rare compared to benign ones, making their detection an imbalanced classification problem. As discussed by \textcolor{red}{insert citation}, an appropriate measure to evaluate SSD methods are false positive and false negative rates as well as the \textit{Area-under-ROC-curve} (AUC) for threshold-based methods.

The false positive rate is defined as

\begin{align*}
\frac{\text{number of benign connections classified as malicious}}{\text{overall number of benign connections}},
\end{align*}

while the false negative rate is defined as

\begin{align*}
\frac{\text{number of malicious connections classified as benign}}{\text{overall number of malicious connections}}.
\end{align*}

Some methods additionally try to predict the length of the stepping stone chain. 


\section{Results}

\subsection{Unperturbed data}




\bibliographystyle{IEEEtranS}
\bibliography{NDSSrefs}

\appendix




 

\end{document}
