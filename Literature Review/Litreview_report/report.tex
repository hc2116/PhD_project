\documentclass[a4paper,12pt,twoside]{report}
\usepackage[left=2.5cm,right=2.5cm,top=2cm,bottom=3cm]{geometry}
%\usepackage[square,authoryear,sort]{natbib}
\usepackage{url}
\usepackage{xcolor}

\begin{document}

\title{\LARGE {\bf Literature Review}\\
 \vspace*{6mm}
}

\author{Henry Clausen}
%\date{October 2008}

\maketitle

\begin{abstract}
Text
\end{abstract}


\chapter{Introduction}

Computer usage can be diverse, and human induced activity on a computer is not constant, but varies in correspondence to the particular task conducted on that computer. In this work, we present a Bayesian framework that models a personal computer's network traffic in order to quantify different states in its usage. For this, we will develop a new hierarchical model based on the \emph{Markov Modulated Poisson Process} that identifies temporal patterns in the arrival of network flow events, and relates them to a latent discrete process which represents the device state. Motivation for this work stems primarily from current interests in cyber-defence, and our inference method is intended to be a critical building block on which a broader cyber-security system would be based. Moreover, this work is heavily related to current procedures in network modelling, for which it might be of future interest.


In the wake of devastating personal information leaks, concerns over cyber-security are at an all-time high. 
Sophisticated data breaches such as the attack on \emph{JP Morgan Chase} in 2014  affect hundreds of million customers and inflicts tremendous financial, reputational, and logistic damage %\citep{walters2014cyber}
. Cyber-security incidents increased by 38\% in 2017, and the global cost of cyber crime is estimated to reach \$2 trillion by 2019 %\citep{conteh2016rise}
. The prevention of cyber crime is therefore a globally demanded necessity.

One reason for the recent rise of cyber crime is the increased use of sophisticated techniques for the attack of specific targets. Attackers use customised social engineering and custom-build malware to pass common security frameworks.
Existing solutions to commercial intrusion detection in computer networks are often based on detecting signatures of previously uncovered  and  analysed  attacks. Examples of such signatures include  file  hashes\footnote{A hash function encodes a file with a basic data structure into a number or string, which is known as the file hash. Every file is uniquely identifiable with its file hash.} of malicious software, blacklisted IP addresses and domain names, and characteristics of known Command-and-Control (C\&C) protocols. Detection of a signature usually indicates an imminent intrusion and triggers investigation.

Adjusting existing attack procedures in order to shed previously identified signatures is simple: A file hash can be altered by minor modifications in the program and IP and domain addresses can be switched by changing servers.
A sophisticated attack will employ new, customized protocols and software that is fitted to the targeted computer infrastructure, and thus will not show any previously identified signatures. 

A different approach to Network Intrusion detection is based on the notion that network intrusions and malicious activity leave different traces in computer event logs\footnote{system calls, network traffic, authentication events, etc.} than normal computer behaviour, and that these differences can be quantified using more data-oriented methods from the area of statics, machine learning, ...

A widely used source of such events are network traffic logs...

In this review, I will focus on methods 

on quantitative models of event logs

the accurate reflection of normal behaviour in a computer network.

A relatively new approach to intrusion detection is based on the accurate reflection of normal behaviour in a computer network. When anomalous behaviour is observed, alternative hypotheses can be formed that reflects attack behaviour. An intrusion is therefore treated as a cumulation of improbable events. Such events can include previously unobserved edges between computers, new processes in combination with E-mail clicks, or failed network logins. To build such a framework, accurate models of several key characteristics of computer network dynamics are necessary in order to capture the aspects that separate regular from irregular behaviour.


\section{Network Intrusion Detection}

Anderson \cite{anderson1980computer} define an intrusion attempt or a threat
as an unauthorized and intentional attempt to either access or manipulate information, or to render a system unreliable or unusable. Such attacks can be very diverse in their nature: They can be used to achieve different goals, and correspondingly exploit different types of tools and vulnerabilities. From a network perspective, usually four classes of malicious traffic are distinguished:

\begin{enumerate}
\item \textit{DoS-attacks}: A denial-of-service attack is an attempt to remove ability of a particular computer to communicate with other machines over an extended period of time. Such attacks are usually targeted at network servers in order to disrupt the service it is providing. All major types of DoS-attacks achieve this by overwhelming the target server with service requests, which are usually corrupted in a way that causes the server to bind resources unnessecarily long for each request, and thus losing its capability to process other requests. SYN-floods for example exploit the TCP protocol by sending many SYN-requests to the server while ignoring the SYN-ACK response packets sent by the server. This causes the server to keep waiting for a response by the attacker and thus binds resources while being computationally very cheap for the attacker.

\item \textit{Network probing/Reconnaissance attack}: The purpose of network probing attacks is to gather information about computers in a network and possibly find vulnerabilities which can be exploited in further attacks. This typically involves sending specific service requests to other computers in the network, and gather information, such as open ports or operating system on a machine, contained in the corresponding response packets.

A common type of network probing attacks is \textit{port scanning}. Its aim is to gather knowledge of computers in the network than run vulnerable services, such as HTTP servers, mail servers, and so on. A port scan achieves this by sending queries to one or more network ports on one or more computers in the network. A computer on which the contacted network port is open will respond to the query and thus reveils himself. A port scan can either be vertical, during many ports on one computer are scanned, or horizontal, where the attacker scans a small number of ports on many computers in the network. 

\item \textit{Access Attacks}: These are attacks that try to gain unauthorized access to 
U2R R2L 

\item \textit{Data Manipulation Attack}: Sometimes used to start access attack, for example by passing malware instead of software update (as Flame did)

\item \textit{C}\&\textit{C traffic}: Botnet (extended communication and continuous), ransomware (limited traffic), any malware that communicates with outside, usually port 80 or 443 detection techniques very different, botnet detection using the fact that multiple hosts are having same behaviour

\item \textit{Data exfiltration}

\end{enumerate}

A lot of successful research has been addressed to detecting botnets and port scanning, so we won't address it in this literature review.

\section{Data Sets}

In order to evaluate its ability to model the behaviour of a network and to identify malicious activity and network intrusions, new methodologies have to be tested using existing data sets of network traffic. We can generally distinguish four different types of data sets containing network traffic:

\begin{enumerate}

\item \textbf{Real network data containing known intrusions}: 

\item \textbf{Real network data containing injected intrusions}:

\item \textbf{Real network data containing no intrusions}:

\item \textbf{Untruthed real network data}:

\item \textbf{Synthetic network data with/without injected intrusions}:

\end{enumerate}

Write about privacy concerns and general problematic of getting data

Furthermore, we can also identify different formats of network data:

\begin{itemize}

\item \textbf{Raw packets}

%\item \textbf{Packet headers}

\item \textbf{Network flows}

\end{itemize}

\subsection{Existing data sets}

\subsubsection*{Los Alamos National Laboratory, 2015 - Comprehensive, Multi-Source Cyber-Security Events \cite{akent-2015-enterprise-data}\cite{kent-2015-cyberdata1}}

In 2015, the Los Alamos National Laboratory (LANL) released a large data set containing \textbf{network flow} traffic from teir corporate computer network, which contains about 17600 computers. The data was gathered over a period of 58 days with about 600 million events per day. The data only contains internal network connections, i.e. no flows going to or coming from computers outside the network are included. IPs and ports were de-identified (with the exception of the most prominent port), but are consistent throughout the data. Since the data stems exclusively from one corporate network, it can be assumed that it shows more homogeneity in the observed traffic patterns than  general network traffic.

Additionally, the data set also contains other event sources which were recorded in parallel in order to give a more comprehensive look at the network, and could be very useful when investigating a detection approach that correlates multiple event sources. These sources include process events and authentication events from Windows-based computers and servers, and DNS lookup events from the DNS servers within the network. 

The dataset furthermore contains a labelled set of redteam events which should resemble intrusions. However, these events are not part of the network flow data and only contain information about the time of the attack and the attacked computer. These events apparently resemble remote \textit{access attacks}, are not described further and appear to be artificial or injected into the data set. It is thus not certain how well they resemble actual network intrusions.


LANL released another data set containing network flow traffic from their network in 2017 \cite{turcotte17}. This data set is similar to the one from 2015, but spans over a longer period of time, 90 days. Furthermore, it contains no labelled malicious activity,  however that does not mean that the data is completely free of malicious activity.

\subsubsection*{CTU 2013 \cite{noauthor_ctu-13_nodate, garcia2014empirical}}

The \textit{Stratosphere Laboratory} in Prague released this dataset in 2013 to study botnet detection. It consists of more than 10 million labelled \textbf{network flows} captured on lab machines for 13 different botnet attack scenarios. Additionally, the raw packets for the botnet activity is also available for attack analysis. 

The labelling in this dataset is different from other datasets as each  flow  in  the  list  is  labelled  based  on  the  source  IP  address.  In  the experiments,  certain  hosts  are  infected  with  a  botnet  and  any  traffic  arising from such a host is labelled as Botnet traffic. Traffic from uninfected hosts is labelled as Normal. All other traffic is Background, as one cannot classify it. 

A criticism of this dataset is the unrealistically high amount of malicious traffic contained in the dataset, which makes it easier to spot it while reducing false positives. Furthermore, the way normal or background traffic is generated is described only poorly and leaves the question how representative it is of actual network traffic.

\subsubsection*{UGR 2016 \cite{macia2018ugr}}

The UGR'16 data set was released by the University of Grenada and contains \textbf{network flow}\footnote{netflow v9} data from a spanish 3-tier ISP. This ISP is a cloud service provider to a number of companies, and thus the data comes from a much less structured network than the LANL data. It contains both client's access to the Internet and traffic from servers hosting a number of services. The data therefore contains a very wide variety of traffic patterns, an advantage emphasised by the authors. IP-adresses are consistently anonymised while network ports are unchanged. However, it is not ensured that the traffic capture is complete, i.e. that all traffic coming from and going to a particular machine is captured.

A main focus in the creation of the data was the consideration of long-term traffic evolution and observable periodicity in order to enable the testing of so called \textit{cyclostationary} traffic models. The data set correspondingly covers a very long period, spanning from March to August of 2016, and containing about 14 GB of traffic per week. 

The data is split into a training set and a test set, with the latter containing labelled attack data. This attack data does not stem from rogue agents but is in part generated in controlled attacks on victim machines, and in part injected from previously observed malware infections. The attack data is therefore does not truly correspond to actual attacks, but achieves a high degree of similarity. The implemented attacks contain:
\begin{itemize}
\item DoS attacks (controlled attacks),
\item Port scanning (controlled attacks),
\item C\&C traffic from a botnet (injected).
\end{itemize}

The authors also acknowledge that the background traffic is not necessarily free from further attacks. In fact, three real attacks have been observed and labelled, corresponding to IP-scanning and a spam mail campaign.

\subsubsection*{UNSW-NB 2015 \cite{moustafa_unsw-nb15:_2015}}

The data set realeased by the \textit{University of New South Wales} in 2015 contains real background traffic and synthetic attack traffic collected at the "Cyber Range Lab of the Australian Centre for Cyber Security". The data is collected from a small number of computers which generate real background traffic, and is overlayed with attack traffic using the \textit{IXIA PerfectStorm tool}. The time span of the collection is in total 31 hours.

An advantage of the collected data set is the inclusion of both \textbf{raw packets} and \textbf{network flows} along with two other data formats containing newly engineered features. This allows a more detailed analysis of the data and possibly a better distinction between attack and benign traffic. In total, the data contains 260 000 events.

Another advantage of the data is the variety of attack data, containing a number of DoS, reconnaissance, and access attacks. However, due to the synthetical injection of these attacks, it is unclear how close they are to real-world attack scenarios.

Since this data set is collected from a relatively small number of machines and during a limited period of time, it is furthermore unclear how suitable for capturing both the temporal evolution and the heterogeneity of real background traffic.

\subsubsection*{CICIDS 2017 \cite{gharib2016evaluation}\cite{sharafaldin2018towards}}

This data set, released by the \textit{Canadian Institute for Cybersecurity} (CIC), contains 5 days of network traffic from 12 computers. These computers all have either different different operating systems such as Windows, OSX, or Ubuntu, or different versions of the same operating system in order to enable a wider range of attack scenarios. The network  furthermore contains switches, routers, a web server, a modem, and a firewall in order to ensure a realistic network topology. The traffic data itself consists of \textbf{labelled benign and attack traffic}, and is available as 11 GB per day of \textbf{raw packets} with payloads, or as \textbf{network flows}. 

It was ensured that the data contains all traffic coming and going from individual machines. However, in contrast to other data sets, the background traffic is not directly generated through user interactions on the machine, but by using a method to profile abstract user behaviour in different traffic protocol. The purpose of this is to make the traffic more heterogenuous and to ensure that different types of behaviour are present in the data during the comparably short time span. This  However, it is not completely clear how much of the underlying structure of real traffic is lost in the process, and therefore how suitable this data is to build models of benign user activity.

The attack data of this dataset is one of the most diverse among NID datasets, as it contains a variety of up-to-date attacks, such as different types of DoS attacks, SQL-injections and Heart-bleed attack, network scanning, or botnet activity. These are not always successful in order to reflect actual attack scenarios. However, the authors did not describe very well how the data from these attacks is generated and combined with the background traffic as it is also processed through a form of profiling engine. 

The CIC released another very similar dataset to this one in 2012.

\subsubsection*{DARPA 1998 \cite{lippmann2000evaluating}}

The \textit{Defense Advanced Research Projects Agency} released the first major dataset to test network intrusion detection systems. The data stems from two experiments at the \textit{MIT Lincoln Laboratory} were multiple victim hosts running Unix and Windows NT were subject of over 200 attacks of 58 different types. The data spans three weeks of training and two weeks of testing data and contains \textit{raw packets} that are labelled. It was since then heavily used as a benchmark to test new detection methods. %In 2000, the original data was post-processed to reduce some of its shortcomings.

Also due to its prominence, it was heavily scrutinised and received a lot of criticism for its lack of realistic background traffic, which was generated through simulation procedure, and the presence of artifacts from these simulations in the data that could heavily skew any model relying on benign traffic. Also, the high percentage of attack traffic in the data is described as unrealistic.

Furthermore, since the dataset is now more than 20 years old, it is remarked that both the benign and attack traffic does not resemble modern network traffic anymore. 

\subsubsection*{KDD Cup 1999 \cite{cup1999data,cup1999dataset}/NSL-KDD 2012 \cite{tavallaee2012nsl}}

The \textit{MIT Lincoln Laboratory} created this dataset in 1999 by processing portions of the 1998 DARPA dataset with new labels for a competition at the conference on \textit{Knowledge Discovery and Data Mining}, and is the most widely used dataset in intrusion detection. It contains 2 million connections summaries in a new format and in total 38 attack types. This new format is essentially a form of \textbf{network flows} with a greatly increased number of features, 46 in total, whichgive additional details to the origin of the connection. Since the KDD'99 data stems directly from the DARPA dataset, it faces the same problems and criticism. 

The \textit{Canadian Institute for Cybersecurity} postprocessed the KDD'99 data in order to address some of its shortcomings. This includes removing redundant records, balancing the size of the training and test data, and adjusting the proportion of attack traffic in the data. However, the biggest criticism from the KDD'99 and the DARPA data, the unrealistic generation of background data, still prevails.

\subsubsection*{LBNL 2013 \cite{pang2005first}}

This dataset released by the \textit{Lawrence Berkeley National Laboratory} in 2005 is the first one to examine internal network traffic inside a modern enterprise. It contains more than 100 hours of \textit{packet headers} from several thousand internal hosts. 

This dataset contains no known attack traffic, and is therefore only suitable for traffic analysis and model fitting analysis. Furthermore, as being the first dataset containing enterprise traffic, privacy concerns caused the authors to remove any possibilities to identify individual IP addresses.

In 2011, Saad et al. \cite{saad2011detecting} combined this dataset with existing botnet traffic to create a dataset containing both benign and attack traffic. 

\subsubsection*{UNIBS 2009\cite{UNIBS2009data}}

This dataset was collected on the campus network of the \textit{University of Brescia} on three consecutive days in 2009. The dataset contains in total 79000 anonymised TCP and UDP \textit{network flows}. 

This dataset is not directed towards intrusion detection research, but was made as \textit{ground truth data} for traffic classification. It therefore contains labels which indicate which of in total six applications generated the corresponding traffic flow. It might however still be of interest for model assessment in intrusion detection that is relying on traffic classification.

\subsubsection*{CAIDA 2016 \cite{walsworth2015caida}}

The \textit{Center for Applied Internet Data Analysis} started collecting network traces from a high-speed backbone link in 2008 with the collection still ongoing. The data is available in anonymised yearly datasets containing one hour of \textbf{packet headers} for each month. 

Since the traffic is collected from a backbone link, it is very unstructured and heterogenuous. It is furthermore not necessarily free from attack traffic. Although this dataset has been used for intrusion detection before, it is more suitable for general internet traffic analysis.


\subsubsection*{MAWI 2000 \cite{sony2000traffic}}



Similarly to the CAIDA dataset, this dataset contains \textbf{packet headers} from the WIDE backbone. It is therefore similarly unstructured, anonymised, and not free from attack traffic. Since this dataset was already collected and released in 2000, it can also be remarked that the contained traffic is too old to represent modern traffic. 

\subsubsection*{ADFA 2013/2014 \cite{creech2014developing,creech2013generation}}

The ADFA datasets, released by the \textit{University of New South Wales}, focuses on attack scenarios on Linux and Windows systems as well as \textbf{stealth attacks}. To create host targets, the authors  installed web servers and database servers , which were then subject to a number of attacks. 

The dataset contains both attack traffic and benign traffic. However, the dataset is directed more towards attack scenario analysis and is criticised as being unsuitable for intrusion detection due to its lack of traffic diversity. Furthermore, the attack traffic is not well separated from the normal one.

\subsubsection*{ICT datasets \cite{USC2010ICT}}

The \textit{Impact Cyber Trust} releases cyber security oriented data. Its repository includes many data sets, synthetic as well as real captures, from different sources. Many datasets focus on observed attack data and thus are not directly applicable to intrusion detection. Furthermore, there is in general very little information provided that describes a dataset's origin, which makes it hard to investigate the network topology.

Among the more useful datasets are the \textit{USC datasets}\footnote{DS-062, LANDER Data, and DS-266}, which contain network traffic (both \textbf{packet headers} and \textbf{network flows}) from academic networks in the US between 2008 and 2010. The datasets are very large, with the largest one covering 48 hours and containing 357 GB of packet headers. 


\chapter{Existing literature}

Existing literature on intrusion detection can be divided into two approaches: 

\begin{itemize}
\item \textbf{Misuse detection}
\item \textbf{Anomaly detection}
\end{itemize} 

In misuse detection, abnormal or malicious behaviour is defined first before developing a model to separate the defined behaviour from other traffic. This approach is often used to detect reoccuring patterns in known intrusion. Applications include botnet detection, where the behaviour of many machines connecting to one (or more) C\&C servers is defined as malicious activity, or port scan detection, or stepping stone/relay detection.

In contrast, anomaly detection aims at building a model of normal system behaviour that is accurate enough to spot any malicious behaviour as traffic that deviates from the estimated model. Anomaly detection is principally more difficult than misuse detection since the traffic model has to incorporate potentially very heterogenuous traffic behaviours. However, it is generally acknowledged that anomaly detectionhas is more suitable to detect new and previously unseen malicious behaviour as it makes no definite assumptions on the anomalous behaviour. 
\linebreak

In reality, anomaly and misuse detection are not necessarily mutually exclusive, and there is a fluent passage between the two. This is because many anomaly detection approaches choose a particular set of features to be modelled with a particular threat in mind. For instance, models for the number of connections of a machine are naturally suitable for detecting DoS attacks, port scans, or Worm attacks. 

\chapter{Anomaly detection}

Anomaly-based intrusion detection moved into the focus of researchers at the end of the 90s, with many advances and new ideas being implemented between 1998 and 2005. 

\subsection{Payload-based approaches}

\subsection{Approaches based on Volume or Traffic Aggregation}

\subsubsection{Entropy-based}



\subsubsection{Subspace projection/PCA-based}

\textbf{Principal Component Analysis} is a statistical form of \textit{orthogonal coordinate transformation} to convert a set of observations (or feature vectors) into a set of linearly independent variables\footnote{the \textit{principal components}}. These variables uncorrelated variables each account for differing amounts of the variation contained in the data. By projecting an observation only onto the components that account for the most variation, it is possible to retain most of the information while operating in much lower dimensions.

\textbf{Lakhina et al.} \cite{lakhina_diagnosing_2004,lakhina_characterization_2004}  introduced a PCA-based anomaly detection method for network traffic in 2004. In their approach, they aggregated the network flows for each OD\footnote{Origin-Destination} pair into 5-minute bins, with the number of transferred bytes, packets, and flows being the features for each bin. Each 120 consecutive bins were then treated as a an observation (with $3\cdot 120$ variables), and PCA was then applied to the collection of observations. The first $5$ principal components are then identified as the dominant temporal patterns. Anomalies were then identified as observations that could only very poorly reconstructed using the first $5$ principle components. Since then, this approach has been adopted to several other datasets without much methodological advances. Camacho et al. \cite{camacho_pca-based_2016} proposed an improvement to the existing PCA-based approach with a more natural implementation of spotting anomalies. 

This approach can be applied to individual OD pairs, or on a network-wide basis by using q-statistics to spot multivariate anomalies. The approach was tested on data from the Abilene backbone network, and worked well to identify significant episodes such as DoS attacks, fast spreading worms and other large-scale scanning activity, alpha-flows, or power outages. 

Naturally, since the traffic is aggregated into bins and the temporal behaviour of these bins are examined, this approach is aimed towards identifying attacks with a comparably large volume of traffic, even if they are isolated in time. It is however unlikely that it is capable of spotting smaller U2R and R2L attacks or C\&C traffic. Another possible criticism is that anomalies are not spotted immediately, but in the worst case after hours.


\textbf{Ringberg and Rexford} \cite{ringberg_sensitivity_2007} described a PCA-based approach to traffic anomaly detection as very sensitive to small differences in the number of used principal components and to the level of aggregation of the traffic measurements. Furthermore, the training data has to be absolutely free of any traffic anomalies, otherwise the projection onto the first principle components can change drastically.


\subsubsection{Wavelet-based}

Wavelet modelling is a frequency-based signal processing approach. Amplitudes of most signals can be described as a finite sum of wavelets with different frequencies. These frequency coefficients can then be used as a measure to describe the signal's generalised behaviour, and to compare with future data from the same signal. Three significant papers applying wavelet modelling to intrusion detection can be found:

%\subsubsection*{Barford et al. - A Signal Analysis of Network Traffic Anomalies, 2002}

Both Barford et al. \cite{barford2002signal}, and Thottan and Ji \cite{thottan2003anomaly} introduced wavelets to network anomaly detection in 2002/2003. Both approaches are fairly simple, as they only look at the network flow numbers from multiple machines at different points in the network, aggregated into 5-minute intervals. Using enough anomaly-free training data\footnote{It is crucial that this data is absolutely free of anomalies that are to be detected}, this volume signal can be described by a set of frequency-components. Barford et al. then compare the frequencies of any future traffic episodes against this set, and marked as anomalous if a treshold is exceed. The approach is directed towards detecting network-wide flash crowds, DoS attacks, and outages, which is evaluated using proprietary data. Thottan and Ji detected anomalies by looking at the reconstruction error of such traffic episodes using the estimated frequency-components instead comparing different estimates. The reconstruction error is assumed to follow a gaussian distribution, and anomalies can be detected using a hypothesis test.

Jian et al. \cite{jiang2014transform} proposed a refined wavelet-based model in 2014 which is applied to individual OD pairs. All observed OD pairs in the network are grouped into $q$\footnote{number depends on network topology} groups. To each group, an S-transformation (a modified version of a wavelet-transformion) is applied. The signal for each OD pair is then reconstructed using only the estimates of the high-frequency components since these correspond to any bursty behaviour. Now, the reconstructed signal is free from any slow variation and contains only fast and bursty behaviour. The assumption of the authors this degenerated signal must be heavily correlated between individual OD pairs. Using a sliding window, the pairwise correlations of each OD pair is computed. If the correlation for any pair falls below a certain treshold, this pair is marked as an anomaly. The approach is designed to detect volume-intensive attacks on busy servers, the exact motivation for this approach however is not described very well by the authors. The evaluation is done using data from the \textit{Abilene backbone}.

It is clear that any approach that looks exclusively at the traffic volume either between individual hosts or in a network-wide fashion will only detect attacks with sufficient attack volume, such as DoS attacks. Additionally, 



\subsection{Event-based}

The majority of network anomaly detection approaches are based on point anomaly detection, in other word they identify individual events as malicious solely on the observed characteristics of this event. Such events are usually either individual network packets or flows. In contrast to approaches on aggregated traffic, which can usually only detect attacks with a certain traffic volume, an event-based approach is independent of the traffic volume and therefore more suitable to identify activities consisting of only a few events, such as data exfiltration, R2L attacks, or C\&C communication.

\subsubsection{Direct application of Machine Learning}

Due to the availability of datasets such as DARPA'98 or KDD'99 which are labelled and rich in both benign and malicious traffic, it is tempting to apply existing machine learning approaches directly onto the event features provided in the data, and there exists a large body of literature doing exactly that. Unfortunately, such approaches often lack the necessary understanding of the data and are overfitting, or do not learn any generalisable behaviour. 


\textcolor{red}{ Also, a number of different classifier techniques were applied on labelled datasets. Although they often achieve good accuracy in the DARPA'98 or KDD'99 datasets, it is unclear how well they translate. Should I include papers using classifiers}


I will briefly discuss some of the better ones here:

%\textbf{Wang and Battiti} proposed an anomaly detection based on 


\subsection{Statistics-based}

\subsubsection*{Mahoney and Chan - Learning Nonstationary Models of Normal Network Traffic for Novel Attacks, 2002 \cite{mahoney2002learning}}

Mahoney and Chen were one of the first to develop statistical methods to identify anomalous events in network traffic. Their approach consists of two separate scoring stages.

The first stage is the \textit{packet header anomaly detection} (PHAD). Here, the 33 different fields of an Ethernet-transmitted packet are converted from their one to four bytes to an integer value. The gathered values for each field are then clustered in a simple agglomerative fashion, and the clusters are updated each time a new packet arrives in order to keep the number of clusters below a treshold. The anomaly score of a packet is then proportional to the number of fields in which the clusters had to be updated. 

At the second stage, the \textit{application layer anomaly detection} (ALAD), scores are asigned to the packet according to a frequency table build using previously collected packets. These frequency tables address the several combinations of a variable conditional on another variable. These variables include source or destination IPs, destination port, TCP flags, or the first word of the payload. An interesting factor considered by the authors is the inclusion of the time since last observance for each of these frequency tables in the anomaly score.

The approach was tested on the DARPA'98 dataset and detected 70 out of 180 attacks while raising 100 false alerts. When the unrealistically high number of malicious packets in the data is considered, the number of false alerts is alarmingly high. Another issue with this publication is that the authors claim that they are building nonstationary models, yet give little how these models should adapt over time. 

\subsubsection*{Kruegel et al., Service Specific Anomaly Detection for Network Intrusion Detection, 2002 \cite{krugel2002service}}

Kruegel et al. developed an approach that is aimed fitting individual models for each of the different services generating network traffic. Their assumption is that by concentrating on only one type of traffic, statistical data with lesser variance can be collected. 

The approach works as following: Once a connection is openened, the packet processing unit reads the first packets of a connection and extracts the specific service, such as a get request for a HTTP request. It is then asigned an anomaly score based on the different aspects: The type of service, the length of the request, and the payload contained in the request. 

The anomaly score associated with the type of service is proportional to the negative logarithm of the service frequency observed in the training data. Thus, rare services receive a higher anomaly score. 

To score the length $l$ of the request, the mean $\mu$ and standard deviations $\sigma$ of request lengths in the training data is estimated using maximum likelihood. The score is then proportional to $(l-\mu)/\sigma$.

Finally, the payload is scored according to a frequency distribution of the letters occuring in the training data. The deviation of a payload from the distribution can easily be estimated using a $\chi^2$ test. By scoring the payload of a service request, the authors hope to detect malicious requests that try to disrupt the reciver through a corrupt combination of non-printable or replaced characters. Kruegel et al. \cite{kruegel2005multi} later greatly improved the payload scoring specifically for HTTP traffic by using a \textit{Markov model}.


In their evaluation, the authors only considered DNS traffic due to lack of resources and space. Testing was done after a calibration of the anomaly tresholds by attacking their own DNS servers with 5 different attacks, all of which have been detected. However, evaluation of other services and on independent data would shed more light on the actual performance. Another important issue not addressed by the authors is the possible temporal drift of the estimated distributions.



\textbf{Both} of these papers introduced introduced new concepts of how to model the distributions of individual event features which take into account the nature of network traffic. However, there is a lot to criticise about these approaches. The developed estimation and scoring methods lack a broader probabilistic foundation and can be improved greatly. Furthermore, these papers do not address any possible interdependence of features, which could lead to serious mismodelling of behaviour observed as anomalous. Also, it is unclear if these models will provide behaviour over time.




\subsection{Clustering based}

\subsection{Representation-learning based}

Representation learning, also called \textit{feature learning} is a set of techniques aimed at automatically learning underlying structures in raw and noisy data, and are in a broader sense a form of density estimation. These techniques are often based on learning lower dimensional representations of the data, similar to subspace-projections, and are therefore suitable for data with highly correlated variables. Existing methods are often based on neural-networks and backpropagation. Learning of normal traffic behaviour can be done directly using representation learning instead of deriving probability distributions and correlations of individual traffic variables first. However, current methods are only suitable for numerical variables and not for categorical ones.



Ramadas and Ostermann \cite{ramadas2003detecting} in 2003 proposed the use of \textbf{self-organizing maps} (SOM) to learn the representation of individual types of network services. A self-organizing map projects input data onto a two-dimensional lattice, which is why they are often used for data visualisation. The projection is learned using groups of competitive neurons. Each generation drops neurons which have different representations from the group, which makes this approach particularly computation-intensive. Since the projected data lies densely together, the authors train the map with normal traffic and detect anomalous events via their distance to the nearest neighbour. The authors however only evaluate their approach using 6 numerical features from DNS and HTTP network flows which they collected themselves. This makes a performance evaluation difficult and also leaves the question open how much knowledge is gained by using only 6 different flow features. Kayacik et al. \cite{kayacik2007hierarchical} later extend this approach to all 41 numerical features of the KDD'99 data. The evaluation showed most success in the detection of DoS and probing attacks.

A more direct approach to outlier detection is provided by Hawkins et al. \cite{hawkins_outlier_2002} in 2002. They applied a \textbf{replicator neural network}\footnote{Also called \textit{Autoencoder network}} to the numerical features of the KDD'99 data. A replicator network tries to accurately reconstruct any input data it receives after sending it through a lower-dimensional bottleneck. The difference to an SOM is that learning is based on error-correction. By training it on normal traffic, the authors build a model that can reconstruct any normal traffic from its lower-dimensional representation with small errors. Anomalies are then detected as input data which is not reconstructed well and therefore deviates substantially from the learned data structures. Supposedly, a replicator network is robust agains small numbers of outliers in the training data. However, it requires careful examination how well this assumption translates onto network traffic. 

Gao et al. \cite{gao_intrusion_2014} use a similar technique called \textbf{deep belief networks} (DBN) on the KDD'99 data. They have a similar structure to replicator networks, but training is more difficult since their hidden layers are probabilistic. The authors mainly focus on explaining the benefits of using probabilistic neurons and discussing possible ways how to train a DBN on network traffic while not providing a thorough discussion of their results.






\subsection{Temporal correlation/Semantics-based}


Noble and Adams \cite{noble_real-time_2018, noble_correlation-based_2016} have recently proposed \textit{ReTiNa}, a tool that measures temporal changes in the correlation between individual events in order to find intrusions on individual hosts. In their approach, they estimate the correlation between the time passed between two events, also called \textit{interarrival time}, of an OD pair and the associated size or number of packets of the involved events. For this, interarrival time and the size/packet number are modelled as a bivariate gaussian distribution, and the covariance matrix is estimated using maximum-likelihood-estimation. The authors use a sophisticated online-estimation method to adapt the estimates to changes in the correlation structure, which can then be identified by comparison to an offline estimate. Anomalies are then identified as a collection of changes happening across multiple OD pairs on one host or in the entire network by simple hypothesis testing, which decreases the false-positive rate. The assumption here is that different OD pairs are independent of each other.


A big advantage of this approach is that it is adaptive and does not need a training phase, i.e. it is not reliant on attack-free training data. The method was tested both on the LANL network flow data as well as internal data from the \textit{Imperial College Academic network}. The method found several anomalies that coincide malicious activity in the network, but a definitive conclusion whether they are related is difficult to make.

Whitehouse, Evangelou and Adams \cite{whitehouse_activity-based_2016} modeling the number of 
network flow and \textit{user authentication} events on individual hosts as a polynomial function of the time and day and its rarity. Anomalies are then identified using Fisher's product test statistic and the reconstruction error. The method was tested on the LANL data using the auth and the flow sources and wa able to identify persistent structures in the data. 


\textcolor{red}{Sketch based}

These  papers  are  similar  in  that  they  both  develop  sketch-based  methods  to  find
changes in a computationally efficient manner, which allows a large number of machines
to  be  monitored  simultaneously,  even  when  traffic  is  arriving  at  a  high  rate.   More
algorithmically, they perform change point detection in cases where there is a very large
number of sequences to be monitored.
The basic problem they are trying to solve is this: we observe (key,value) pairs, where
the value counts the number of times the key has occurred.  We wish to detect changes
in the rate at which the keys arrive (which can correspond to an increase in the number
of TCP packets, portscans, etc).  These papers work in discrete time using windowing –
for each window of size W, changes in a single sequence can be detected by counting the
number of times the key appears in the window, and comparing this to the forecasted
value, based on all the previous windows and some forecasting model (e.g.  ARIMA, Holt-
Winters, etc).  If the difference between the empirical value and the forecast exceeds some
threshold, a change is flagged.  The question is then how to scale this up so that many
different streams can be simultaneously monitored.  Their solution involves sketches.
The sketch they use is an extension of the Count Sketch.  They have H hash tables
each containing K bins, where K is much smaller than N (N = the number of keys that are
being monitored).  Each hash table is associated with a unique hash function.  Whenever
a key is received, it is hashed using each of the H functions, and the corresponding bin
in  each  hash  table  is  incremented.   Based  on  this,  they  propose  an  estimator  of  the
count (value) associated each key which is unbiased, with variance proportion to (K-1).
However, the variance they get is only a bound, not an exact number.  Similarly, they
propose an estimator of the sum of squares of all elements in the sketch, which is used
for change detection.
The change detection is carried out by simply checking whether the forecast error is
above T * sqrt(sketch variance), for some threshold T..  Because of the inaccuracy in the
count estimator, combined with their forecasting model, they cannot produce a p-value.
(which also prevents them from doing anything more sophisticated like False Discovery
Rate control).
The Krishnamurthy paper has a serious problem related to the Count Sketch that
it uses:  since they do not log the values of the keys themselves (as there are too many,
and doing so would remove the efficiency gains from the sketch), they are only capable
of  saying whether there  has been  a  change  point  in  the sequence  of  counts  associated
with at least one key, but they cannot recover which particular key has changed.  The
Schweller paper introduces a new reversible sketch method to fix this problem, and allow
the changed keys to be recovered.  Their insight is based on the fact that only the sketch
updating needs to performed fast,  in a streaming manner.  The change detection part
on the other hand can be performed offline, and perhaps only needs to be run every few
seconds.  This is essentially an example of stream-cloud integration.
Note the original Count-Sketch paper (“Finding Frequent Items in Data Streams”)
seems to have more detailed theoretical results.
My comment:  as future work, is it possible to get p-values?  This involves compen-
sating for the inaccuracy in the sketch counting, and would make it possible to use more
sophisticated change detection methods.



\subsubsection{Semantic-based approaches using different data sources}



\appendix
% appendices come here

%\addcontentsline{toc}{chapter}{Bibliography}
\bibliographystyle{abbrv}
\bibliography{refs}

\end{document}