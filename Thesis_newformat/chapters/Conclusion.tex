\section{Result conclusions}

\section{Contributions}

\subsection{Microstructure results}

\subsection{Traffic generation paradigm}

\subsection{Anomaly detection}


\section{Central Research questions}\label{Sec:Reseach_questions}

\begin{rquestion}\ \\ 
How well-structured is the space of microstructures observed in the traffic of a machine or a network? To what degree are these microstructures a result of specific computational activities that are of interest for traffic classification and intrusion detection, and how much are they affected by other external variables?
%does noise or input variation blur the observable contextual differences between clearly distinct actions?

%\begin{enumerate}
%\item How can we scientifically quantify closeness between individual actions, and does it translate into the similarity of corresponding traffic structures? 
%\end{enumerate}
\end{rquestion}



\begin{rquestion}\ \\
To what degree can relevant microstructures in network traffic be captured in a model from a training dataset, and how can we achieve this? How can a model adapt to changes of structures in benign traffic?
%\begin{enumerate}
%\item Which parts of network traffic both contain important information and can be represented by a model?
%\item Can be combine models that act at different traffic levels to enhance the amount of context given by the data? 
%\item Can we efficiently disentangle overlaying network flows to isolate otherwise distorted flow groups corresponding to similar actions? 
%\item What is an efficient and realistic way to incorporate other data sources into the modelling procedure? How can this input enhance the learning process and the representation detail of a traffic model? 
%\item How can a model adapt to changes of normal contextual structures?
%\end{enumerate}
\end{rquestion}

\begin{rquestion}\ \\
What is a meaningful representation of traffic microstructures? What requirements must a labelled traffic generation framework fulfill to provide realistic data?
%\begin{enumerate}
%\item What requirements must a labelled traffic generation framework fulfill to provide realistic data?
%\item Can we evaluate the traffic representation of a model through its ability to identify contextual closeness between traffic instances correctly?
%\item How can we quantify the capability of a given traffic model to identify new computational actions on a machine?
%\end{enumerate}
\end{rquestion}


\begin{rquestion}\ \\
What will a contextual model be able to prevent? 
%\begin{enumerate}
%\item What kind of attacks will necessarily show contextual anomalies, and which will not?
%\item Can an adversary adapt his attacks to avoid detection? How can we prevent this?
%\end{enumerate}
\end{rquestion}

\section{Critical analysis of results}

\section{Future work}

